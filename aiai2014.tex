% This is LLNCS.DEM the demonstration file of
% the LaTeX macro package from Springer-Verlag
% for Lecture Notes in Computer Science,
% version 2.4 for LaTeX2e as of 16. April 2010
%
\documentclass{llncs}
%
\usepackage{makeidx}  % allows for indexgeneration
%
\begin{document}
%
%
\mainmatter              % start of the contributions
%
\title{A Machine Learning Application that Matters}
%
\titlerunning{ML Application that Matters}  % abbreviated title (for running head)
%                                     also used for the TOC unless
%                                     \toctitle is used
%
\author{Geoff Holmes\inst{1} \and Dale Fletcher\inst{2} \and Peter Reutemann\inst{3} \and Martijn van Oostrum\inst{4}}
%
\authorrunning{Holmes et al.} % abbreviated author list (for running head)
%
%%%% list of authors for the TOC (use if author list has to be modified)
\tocauthor{Geoff Holmes, Dale Fletcher, Peter Reutemann}
%
\institute{
University of Waikato, Hamilton, NZ, \email{geoff@waikato.ac.nz}
\and
University of Waikato, Hamilton, NZ, \email{dale@waikato.ac.nz}
\and
University of Waikato, Hamilton, NZ, \email{fracpete@waikato.ac.nz}
\and
BLGG AgroXpertus, Wageningen, NL, \email{martijn.vanoostrum@blgg.agroxpertus.com}
}

\maketitle              % typeset the title of the contribution

\begin{abstract}
TODO
\end{abstract}

%%%%%%%%%%%%%%%%
% Introduction %
%%%%%%%%%%%%%%%%
\section{Introduction}
\begin{itemize}
  \item Machine learning that matters?? See Wagstaff’s paper (citations) \\
    http://scholar.google.com/scholar?oi=bibs\&hl=en\&cites=2774690908240883628
  \item Martijn - business side
\end{itemize}

%%%%%%%%%
% ADAMS %
%%%%%%%%%
\section{ADAMS}
ADAMS, the Advanced Data mining and Machine learning System, is a modular,
scientifc workflow engine written in Java. Currently available modules
include support for Weka, MOA, R, image processing (ImageJ, JAI, ImageMagick,
Gnuplot), PDF management and display, spreadsheet manipulation (CSV,
Gnumeric, Excel, ODF), scripting (Groovy, Jython), GIS support
(OpenStreetMap), Twitter, time-series analysis, network support (SSH,
SCP, SFTP/FTP, Email), XML/HTML/JSON processing facilities and
webservice capabilities.

In contrast to current versions of other workflow engines, like Kepler, KNIME
or RapidMiner, where the user places operators on a canvas and connects them
manually, ADAMS organizes the operators (or “actors”) of a workflow
automatically in a tree structure without explicit connections. Instead, so
called “control” actors determine how data flows between actors. Examples of
control actors are, e.g., Sequence, Branch, Tee, Trigger, IfThenElse and
Switch. Apart from “flow control”, there are two further aspects to an actor:
“functional” (primitive actor or manages nested actors) and “procedural” in
terms of input/output of data (standalone, source, transformer, sink).

Using a tree layout has advantages and disadvantages. In terms of advantages,
the layout is very compact, it scales to thousands of actors, avoids
having to manually rearrange the workflow in order to include additional
actors (disconnect/reconnect), is context aware when adding actors (data
types of input and output limit what actors can be inserted) and has
customizable rules for suggesting actors depending on context for common
sequences of actors. Disadvantages are, the tree layout is less intuitive
compared to a canvas-based approach and it only supports 1-to-n connections.
The 1-to-n limitation is mitigated using call-able actors (multiple actors can
channel data into a single actor using its name), containers for storing
multiple outputs, variables and internal storage (re-using data in multiple
locations). Variables can be either used in expressions, e.g., ones for
evaluating mathematical formulas, or attached to parameters of operators. The
latter allows for influencing the flow execution, e.g., for the turning on/off
of sub-flows or dynamically changing the setup of a learning algorithm. The
scope of variables and internal storage can be limited using the LocalScope
control actor.

Some further feature highlights: Though ADAMS is a data-driven workflow,
transporting the data in so called “tokens”, by design rather than an
event-based one, i.e., actors get executed if there is data available for them
to process, it is also possible to trigger sub-flows using cronjobs.  This
allows, for instance, for recurring clean-up operations. Interactive actors are
very useful for developing workflow applications. These actors either prompt
the user to enter or select a value, controlling sub-flow execution, or require
the user to inspect data, e.g., visual inspection of images, influencing the
data flow. By supporting scripting (Groovy/Jython), it is possible to quickly
prototype new actors without the need of compiling Java code and restarting the
workflow application.  Once a workflow has been developed, it is not necessary
to use the graphical user interface for executing it, the command-line can be
used as well (e.g., for use in a headless server environment). Java-code
generation from existing flows is possible as well.

TODO screenshots, simple example


%%%%%%%%%%%%%%%
% AgroXpertus %
%%%%%%%%%%%%%%%
\section{AgroXpertus}
ADAMS - meta-flows, S2000

%%%%%%%%%%%%%%
% Discussion %
%%%%%%%%%%%%%%
\section{Discussion}
\begin{itemize}
  \item Cost-benefit analysis (Martijn)
  \item Number of predictions (Martijn)
  \item Percentages going to wet chemistry (Martijn)
  \item Six Impact Challenges: \\
    2. \$100M saved through improved decision making provided by an ML system $\rightarrow$ S2000
  \item Lack of Follow-through \\
	ADAMS $\rightarrow$ easily integrate ML system into business processes (opposed to plain Weka)

\end{itemize}

%%%%%%%%%%%%%%
% Conclusion %
%%%%%%%%%%%%%%
\section{Conclusion}
ADAMS matters!

%%%%%%%%%%%%%%%%
% Bibliography %
%%%%%%%%%%%%%%%%
\begin{thebibliography}{5}

\bibitem{wagstaff2012}
Kiri L. Wagstaff (2012):
Machine Learning that Matters.
Proceedings of the Twenty-Ninth International Conference on Machine Learning (ICML), p. 529-536, 2012.

\bibitem{reutemann2012}
Peter Reutemann and Joaquin Vanschoren (2012):
Scientific Workflow Management with ADAMS.
Proceedings of the Machine Learning and Knowledge Discovery in Databases (ECML-PKDD), Part II, LNCS 7524, 2012, pp 833-837, Bristol, UK, 2012.

\end{thebibliography}

\end{document}
